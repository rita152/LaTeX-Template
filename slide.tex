\documentclass{beamer}
\usepackage{ctex, hyperref}
\usepackage[T1]{fontenc}

% other packages
\usepackage{latexsym,amsmath,xcolor,multicol,booktabs,calligra}
\usepackage{graphicx,pstricks,listings,stackengine}

\author{张鹏}
\title{组会汇报}
\subtitle{\today}
\institute{厦门大学航空航天学院}
\date{}
\usepackage{XMU}

% defs
\def\cmd#1{\texttt{\color{red}\footnotesize $\backslash$#1}}
\def\env#1{\texttt{\color{blue}\footnotesize #1}}
\definecolor{deepblue}{rgb}{0,0,0.5}
\definecolor{deepred}{rgb}{0.6,0,0}
\definecolor{deepgreen}{rgb}{0,0.5,0}
\definecolor{halfgray}{gray}{0.55}

\lstset{
    basicstyle=\ttfamily\small,
    keywordstyle=\bfseries\color{deepblue},
    emphstyle=\ttfamily\color{deepred},
    stringstyle=\color{deepgreen},
    numbers=left,
    numberstyle=\small\color{halfgray},
    rulesepcolor=\color{red!20!green!20!blue!20},
    frame=shadowbox,
}

\begin{document}

\kaishu
\begin{frame}
    \titlepage
    \begin{figure}[htpb]
        \centering
        \includegraphics[width=0.25\linewidth]{pic/xmu-logo.pdf}
    \end{figure}
\end{frame}

\begin{frame}
    \tableofcontents[sectionstyle=show,subsectionstyle=show/shaded/hide,subsubsectionstyle=show/shaded/hide]
\end{frame}

\section{课题背景与目标}

\subsection{行业动机}
\begin{frame}{课题背景:行业动机}
    \begin{itemize}
        \item 智慧安防、智能医疗等应用场景需要可靠的行为理解能力。
        \item 多模态识别能够综合不同传感器的信息,提高对复杂行为的捕捉能力。
    \end{itemize}
\end{frame}

\subsection{当前痛点}
\begin{frame}{课题背景:当前痛点}
    \begin{itemize}
        \item 现有方案在跨场景泛化和轻量部署方面表现不足。
        \item 模型缺乏可解释且鲁棒的特征融合策略,难以适应真实业务。
    \end{itemize}
\end{frame}

\subsection{项目要求}
\begin{frame}{课题背景:项目要求}
    \begin{itemize}
        \item XMU 航空航天学院承担的纵向项目要求在 2024 年底前完成核心算法的工程化验证。
        \item 需同步考虑算法指标、部署效率与软硬件协同,形成可落地方案。
    \end{itemize}
\end{frame}

\subsection{阶段目标}
\begin{frame}{研究目标:阶段目标}
    \begin{block}{阶段目标}
        \begin{itemize}
            \item 构建覆盖视觉、语音、惯导传感的多模态数据集与基准管线。
            \item 设计鲁棒的表征学习与跨模态对齐策略,支撑复杂行为识别。
            \item 打通端到端推理流程,实现 30 FPS 以上的实时识别能力。
        \end{itemize}
    \end{block}
\end{frame}

\subsection{评估指标}
\begin{frame}{研究目标:评估指标}
    \begin{exampleblock}{评估指标}
        \begin{itemize}
            \item 精度:Top-1 准确率 $\geq 88\%$,F1-score $\geq 0.85$。
            \item 性能:推理延迟 $\leq 33\text{ ms}$,模型大小 $\leq 120$ MB。
            \item 部署:支持 Jetson Orin 与 X86 GPU 的双平台运行。
        \end{itemize}
    \end{exampleblock}
\end{frame}

\section{数据与方法}

\subsection{数据来源与采集}
\begin{frame}{数据来源与采集}
    \begin{table}[h]
        \centering
        \begin{tabular}{lccc}
            \toprule
            数据模态 & 采集频率 & 样本量 & 备注 \\\midrule
            RGB 视频 & 30 FPS & 1\,200 Clips & 室内/户外混合 \\
            IMU 传感 & 200 Hz & 3\,600 序列 & 穿戴式采集 \\
            音频 & 16 kHz & 1\,800 片段 & 附带噪声标签 \\\bottomrule
        \end{tabular}
        \caption{本阶段已完成的数据采集与清洗情况}
    \end{table}
    \begin{itemize}
        \item 数据标注采用三人交叉审核机制,保障事件级别一致性。
        \item 通过自动脚本生成时间戳对齐文件,方便后续融合。
    \end{itemize}
\end{frame}

\subsection{系统方案概览}
\begin{frame}{系统方案概览}
    \begin{columns}
        \column{0.55\textwidth}
        \begin{itemize}
            \item 视觉分支:基于 Video Swin Transformer,支持剪枝与蒸馏。
            \item 语音分支:采用 Conformer 编码器,与视觉特征共享注意力。
            \item 传感器分支:轻量级 TCN,强调时间依赖建模。
            \item 融合策略:多尺度对齐 + 动态门控融合,配合多任务损失。
        \end{itemize}
        \column{0.4\textwidth}
        \begin{figure}[htbp]
            \centering
            \includegraphics[width=\linewidth]{pic/dtmf.pdf}
            \caption{流程示意:数据同步 $\rightarrow$ 特征提取 $\rightarrow$ 跨模态融合 $\rightarrow$ 行为预测}
        \end{figure}
    \end{columns}
\end{frame}

\section{实验设计与结果}

\subsection{实验设置}
\begin{frame}{实验设置}
    \begin{itemize}
        \item 训练环境:8 $\times$ A100,AdamW 优化器,Cosine 学习率,batch=64。
        \item 数据增强:视觉使用 RandAugment+Mixup,音频使用 SpecAugment,IMU 使用抖动/裁剪。
        \item 评价协议:5 折交叉验证 + 跨场景留一验证,确保泛化性。
        \item 发布两条对照实验:单模态基线、全模态融合。
    \end{itemize}
\end{frame}

\subsection{性能对比}
\begin{frame}{性能对比}
    \begin{table}[h]
        \centering
        \begin{tabular}{lccc}
            \toprule
            方法 & Top-1(\%) & F1-score & 时延(ms) \\\midrule
            视觉单模态 & 80.6 & 0.74 & 21.3 \\
            语音+IMU & 82.1 & 0.77 & 18.9 \\
            全模态融合(当前) & \textbf{87.9} & \textbf{0.84} & 28.5 \\
            目标(2024Q4) & 88.0+ & 0.85+ & 33 以下\\\bottomrule
        \end{tabular}
        \caption{关键结果对比,融合模型已逼近项目考核指标}
    \end{table}
    \begin{itemize}
        \item 视觉分支限制了总体表现,后续需进一步蒸馏与剪枝。
        \item 推理延迟仍受多模态对齐开销影响,需要算子优化。
    \end{itemize}
\end{frame}

\subsection{可视化与案例}
\begin{frame}{可视化与案例}
    \begin{columns}
        \column{0.45\textwidth}
        \begin{block}{典型案例}
            \begin{itemize}
                \item 多人协同:融合模型在“交接物品”场景准确率提升 12\%.
                \item 异常行为:通过 IMU 通道捕获微动信息,降低误报。
                \item 噪声环境:语音分支结合噪声标签提升鲁棒性。
            \end{itemize}
        \end{block}
        \column{0.45\textwidth}
        \begin{figure}[htbp]
            \centering
            \includegraphics[width=\linewidth]{pic/xmu-logo.pdf}
            \caption{示意:可根据实际需求替换为行为可视化截图}
        \end{figure}
    \end{columns}
\end{frame}

\section{问题与讨论}

\subsection{挑战与风险}
\begin{frame}{挑战与风险}
    \begin{itemize}
        \item \textbf{数据问题}:户外场景样本不足,需补采夜间与极端光照数据。
        \item \textbf{算法问题}:跨模态对齐仍然依赖手工超参,自动调度策略缺失。
        \item \textbf{工程问题}:Jetson 端推理频率不稳定,TensorRT 插件仍需优化。
        \item \textbf{协同问题}:生态合作方接口迟迟未定,影响整合测试。
    \end{itemize}
\end{frame}

\subsection{讨论要点}
\begin{frame}{讨论要点}
    \begin{columns}
        \column{0.48\textwidth}
        \begin{block}{需要决策}
            \begin{itemize}
                \item 是否投入额外预算扩充夜间采集?
                \item 模型蒸馏优先支持视觉还是 IMU 分支?
                \item 平台优化优先 Jetson 还是 X86?
            \end{itemize}
        \end{block}
        \column{0.48\textwidth}
        \begin{exampleblock}{拟定支持}
            \begin{itemize}
                \item 申请学院 GPU 集群的额外夜间使用时段。
                \item 与合作院校共享噪声标注工具链。
                \item 争取工程团队协助推理框架合并。
            \end{itemize}
        \end{exampleblock}
    \end{columns}
\end{frame}

\section{下一步计划}

\subsection{里程碑安排}
\begin{frame}{里程碑安排}
    \begin{itemize}
        \item 6 月:完成夜间与极端场景数据补采,固化清洗规范。
        \item 7 月:实现跨模态自适应对齐模块,完成蒸馏实验。
        \item 8 月:交付 Jetson 优化版本,时延压缩到 25 ms 以内。
        \item 9 月:联合生态合作方开展系统联调和用户测试。
        \item 10-11 月:撰写论文与技术报告,准备验收材料。
    \end{itemize}
\end{frame}

\subsection{本周行动项}
\begin{frame}{本周行动项}
    \begin{columns}
        \column{0.5\textwidth}
        \begin{itemize}
            \item 数据组:推进夜间场景采集脚本开发。
            \item 算法组:完成多模态门控网络初版上线。
            \item 工程组:定位 TensorRT 插件的内存泄漏。
        \end{itemize}
        \column{0.45\textwidth}
        \begin{block}{风险缓解}
            \begin{itemize}
                \item 预留 2 天回归窗口验证性能波动。
                \item 维护实验记录与配置,避免重现实验困难。
            \end{itemize}
        \end{block}
    \end{columns}
\end{frame}

\section{参考文献}

\subsection{参考文献}
\begin{frame}[allowframebreaks]
    \bibliography{ref}
    \bibliographystyle{xmu}
\end{frame}

\section*{致谢}
\begin{frame}
    \begin{center}
        {\Huge\calligra Thanks!}
    \end{center}
\end{frame}

\end{document}
